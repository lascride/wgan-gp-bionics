\chapter{边界层效应与临界指标的寻找}

在前两章中,我们证明了,$\mathbb{T}_{L}\times [a,b]$上的线性化Euler方程,当初值的边值非零/取零时,分别有$H^{3/2-},H^{5/2-}$稳定性,并在$H^s(s>\frac{3}{2}, \frac{5}{2})$时发生爆破。但之前的讨论并没有给出当$y$靠近边界时,解的奇异性的具体刻画,也没有说明为什么能够直接“猜出”临界指标是$\frac{3}{2},\frac{5}{2}$. 那么本章就从估计边界效应项$H^{(1)}$在$y\rightarrow 0$时的贡献来说明这两个问题。下面我们不妨假设$w_0|_{y=0,1}\neq 0$(等于0时的讨论是同理的),来考虑$\partial_y W$的奇异性.

我们考虑的方程是:
\begin{equation}
\begin{cases}
\begin{aligned}
\partial_t\partial_y W&=\frac{if}{k}(\Phi^{(1)}+H^{(1)})+\frac{if'\Phi}{k},\\
(-1+(g(\frac{\partial_y}{k}-it))^2)\Phi^{(1)}&=\partial_y W+[(g(\frac{\partial_y}{k}-it))^2,\partial_y]\Phi, \\
\Phi^{(1)}|_{y=0,1}&=0. \\
H^{(1)}&=\partial_y\Phi-\Phi^{(1)}, \\
(-1+(g(\frac{\partial_y}{k}-it))^2)H^{(1)}&=0, \\
H^{(1)}|_{y=0,1}&=\partial_y\Phi|_{y=0,1}, \\
(k,y,t)&\in L(\mathbb{Z}-\{0\})\times [0,1] \times\mathbb{R}.
\end{aligned}
\end{cases}
\end{equation}

\section{边界效应带来$\log(t)$的奇异性}
边界项$H^{(1)}$对$\partial_y W$的贡献,自然是用$$\frac{if(y)}{k}\int_0^T H^{(1)}(t,y)dt$$在$T\rightarrow \infty$时的行为来刻画. 

在之前的章节,我们已经证明$$H^{(1)}(t,y)=H^{(1)}(t,0)e^{ikty}u_1(0,y)+H^{(1)}(t,1)e^{ikt(y-1)}u_2(y).$$ 那么积分就有
\begin{equation}
\forall y\in [0,1], \int_1^{T}H^{(1)}(t,y)=\int_1^T H^{(1)}(t,0)e^{ikty}u_1(0,y)+H^{(1)} (t,1)e^{ikt(y-1)}u_2(y)dt.
\end{equation}

在第四章我们证明了$$H^{(1)}(t,0)=\frac{1}{t}\frac{w_0|_{y=0}}{g^2(0)}+O(t^{-s-1}).$$ 因此当$y=0$时,直接积分就有
$$\int_1^T H^{(1)}(t,0)e^{ikty}dt|_{y=0}\gtrsim\log T.$$

下面的引理给出了$y>0$时 (6.2)式的行为刻画.

\begin{lem}
\songti\rm 设$k\in\mathbb{Z}_+$. 则:

(1)当$y\geq \frac{1}{2k}$时,积分$$\int_1^T e^{ikty}{t}dt$$关于$T,k,y$一致有界;

(2)当$0<y<\frac{1}{2k}$时,$$\left|\int_1^T\frac{e^{ikty}}{t}d\right|t\gtrsim\min\{\log T,-\log(ky)\}+O(1);$$

(3)当$0<y<\frac{1}{2kT}$时,$$Re\int_1^T\frac{e^{ikty}}{t} dt\gtrsim \log T+O(1);$$

(4)$T\rightarrow\infty,0<y<\frac{1}{k}$时,$$\left|\int_1^\infty\frac{e^{ikty}}{t}dt\right|\gtrsim-\log(ky)+O(1).$$
\end{lem}

\textbf{证明:}只用证明(1),(2),后面的都是推论。

(1)的证明非常简单,作变量替换$\tau=kyt$即有$$\int_1^T e^{ikty}{t}dt=\int_{ky}^{kyT}\frac{e^{i\tau}}{\tau}d\tau.$$

注意到对固定的$\frac{1}{2}\leq x_1\leq x_2$, 成立$$\int_{x_1}^{x_2}\frac{e^{i\tau}}{\tau}d\tau\lesssim 2.$$

于是令$x_1=ky$即可.

(2)的证明也很简单,在上式中取$x_1=1,x_2=kyT$, 可得$$\int_1^{kyT}\frac{e^{i\tau}}{\tau}d\tau\lesssim 1.$$

再估计$$\int_{ky}^{\min{1,kyT}}\frac{e^{i\tau}}{\tau}d\tau.$$这时注意到$\tau\in(0,1)\Rightarrow 0\leq\cos 1<\cos\tau=Re(e^{i\tau})\leq 1$

所以$$\int_{ky}^{\min\{1,kyT\}}\frac{e^{i\tau}}{\tau}d\tau\dot{=}\int_{ky}^{\min\{1,kyT\}}\frac{d\tau}{\tau}=\min\{\log T,-\log(ky)\}.$$

\begin{flushright}
$\square$
\end{flushright}

\section{临界指标的寻找}

注意到$1\leq p<\infty$时$\log x\in L^p[0,1]$, 所以在$L^p(p\neq\infty)$中不会出现爆破。然而我们的稳定性是在Sobolev空间中考虑的,所以我们希望刻画$\partial_y^s\partial_yW$的长时间行为,并找出$s$取何值时,对应的积分出现爆破。

我们考虑
\begin{equation}
C_s(T,y)=\int_1^Tt^s\frac{e^{ikty}}{t}dt,~~0<s<1
\end{equation}

(6.3)式的积分,不仅形式上可以作为$\partial_yW, \partial_y^2W$之间的插值情况,而且是可以具体计算出来的。准确地说,用Sobolev空间的差商刻画(见附录),的确可以直接算出来,待估计的项就是(6.3)式。

$0<s<1$时,如下引理刻画了(6.3)的行为.

\begin{lem}\songti\rm

(1)
\[
C_s(T,0)=\frac{T^s-1}{s};
\]

(2)当$0<y<\frac{1}{2k}$时,成立
\[
C_s(T,y)\lesssim\min\{T^s,\frac{1}{(ky)^s}\}+O(1);
\]

(3)当$0<y<\frac{1}{2kT}$时,成立
\[
Re(C_s(T,y))\gtrsim\frac{T^s-1}{s}+O(1);
\]

(4)$T\rightarrow\infty$时,存在复常数$c$, 满足
\[
C_s(\infty,y)=\frac{c}{(ky)^s}+O(1).
\]
\end{lem}

\begin{flushright}
$\square$
\end{flushright}

那么由上述引理的(4)就知道,$y^s\in L^p[0,1]\Leftrightarrow sp<1$. 于是$p=2$时,对应的$s=\frac{1}{2}$. 即$\partial_y^{3/2}W$是临界阶数的导数。所以初值的边值非零时,就有$s=3/2$作为临界指标。