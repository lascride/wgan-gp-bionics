\begin{cnabstract}
结构仿生旨在利用生物的结构特征指导工业工程中的产品设计,以达到节约原材料,优化力学性能等目的。3D打印技术的发展大大拓展了结构设计的空间,但从物种中抽象出仿生结构的过程仍然缺乏普适的理论指导。现有的抽象方法主要依赖于相关专家的主观意见,不同方面的仿生需要不同的专业知识,人力成本过高。且主观性太强,不能保证充分利用生物的进化成果。本文提出了一种新的抽象仿生结构的方法,利用生成对抗网络学习生物的结构特征,再基于人为限制产生符合要求的仿生结构。

\keywords{结构仿生, 生成对抗网络, DCGAN, WGAN, iGAN}
\end{cnabstract}
\begin{enabstract}
In this thesis we discuss about the asymptotic stability in Sobolev spaces (mainly about linear stability) and inviscid damping of Euler's equation evolving around the monotonic shear flow $(U(y),0)$ in certain 2D region ($\mathbb{T}_L\times\mathbb{R}$, $\mathbb{T}_L\times [a,b]$). 

Specifically speaking, we prove $H^s$ linear stability in $\mathbb{T}_L\times \mathbb{R}$. As for the case $\mathbb{T}_L\times [a,b]$, we prove the critical spaces for stability/blow-up of linearized 2D Euler are $H^{3/2-}, H^{5/2-}$, respectively for nonvanishing/vanishing initial data on boundary. In the proof of stability results, we use Constant-Coefficient equations as "Toy Model" to do estimates. Based on the stability result, we derive the linear inviscid damping as the long time behaviour of the solution to linearized 2D Euler. As corollaries we derive the scattering results by Du'Hamel's principle. Then we illustrate how we determine the critical index for blow-up. Finally we give the control of nonliearity and discuss the consistency between nonlinear stability and linear stability. Also we illustrate our core mechanism "trading regularity for decaying".

\enkeywords{Euler's Equation, Monotonic Shear Flow, Asymptotic Stability, Blow-up, Inviscid Damping, toy model.}
\end{enabstract}
