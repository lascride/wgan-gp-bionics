\begin{cnabstract}
结构仿生旨在利用生物的各种结构特征指导工程中的产品设计,以达到节约原材料,优化力学性能等目的。3D打印技术的发展大大拓展了仿生结构设计的空间,但从物种中抽象出仿生结构的过程仍然缺乏普适的方法。现有的抽象方法主要依赖于相关专家的专业知识。不同物种,不同用途的仿生需要不同的专业知识,人力成本过高,且主观性太强,不能保证充分利用生物的进化成果。本文提出了一种全新的抽象仿生结构的方法,基于生成对抗网络学习生物的结构特征,并自动生成符合设计要求的仿生结构。

\keywords{结构仿生, 深度学习, 生成对抗网络.}
\end{cnabstract}
\begin{enabstract}
The lightweight and high mechanical performance of biological structures in nature are valuable and inexhaustible resources for lightweight design and make structural bionics an important method in engineering. The development of Additive Manufacturing has greatly improved the functionality and flexibility of structural bionic design, however, people still lack a general guideline to execute the abstraction of biological structures. Current abstraction methods need specific expertise according to the biological structures and the applications and thus are costly. Also, the abstraction is too subjective to take full advantage of creature's evolution. In this paper, we introduce a brand new structural bionic design method based on Generative Adversarial Networks. Our method can automatically generate desired structures learned from nature and thus overcome the shortcomings above.

\enkeywords{Structural Bionics, Deep Learning, Generative Adversarial Networks (GAN).}
\end{enabstract}
