
\begin{thanks}

在中科大数院的四年生活即将画上句号之际,我有太多的人想要感谢。

首先,感谢四年来数院对我的培养。尽管回忆并不都是甜蜜的,但我并不后悔来到了这里:因为这四年我对数学热爱愈发强烈,学习数学的信念愈发坚定。在人生宝贵的本科四年中,我来到了这个也许是最正确的地方。

我非常感谢我的导师赵立丰副教授。从我大三跟赵老师做大研开始,正是在赵老师的指导下,我逐渐对PDE的调和分析方法产生了兴趣,并自发地去学习更多有关的内容。赵老师高屋建瓴的观点、严谨求实的态度都将使我终身受益。我还要感谢殷浩副教授。担任殷老师的助教期间,我从殷老师的课上学到了很多精细的技术和一针见血的思路。而这正是数学研究必需的。有这一年的助教经历,我感到十分幸运。

我要特别感谢史济怀老先生,数分课上那句难忘的教诲“虽然你们以后会把这些忘掉,但学过和没学过的不一样的”使我受益终生。此外,感谢薄立军老师、任广斌老师为我申请出国读博士写推荐信。感谢王毅、王作勤、刘党政、刘世平、宋百林、盛茂、胡治水等老师,听各位老师讲课是一种享受。

我身边还有很多帮助过我的同学。我尤其要感谢黎泽、罗翔、刘彦麟、廖羽晨等师兄,和你们的交谈使得我尽早接触到了更深刻、更前沿的知识。感谢同班的南昌老乡胡家昊同学,大二相识至今,我们在数学和生活上有了越来越多的共同语言:面对问题我们共同解决,面对琐事我们相互倾诉。你是我身边最值得信赖的朋友,与你相识是不可多得的幸事。此外,我还要感谢陶睿、李弢、王梓安、马翘楚、马明辉、姚东、汪子琦、林展、姜翰生等同班同学,你们在学习和生活上给予了我很多帮助。感谢丘赛队友马骁、李震昊、汪亦桐。感谢毛天乐、罗宇杰、曲彦霖、张桐、张羽丰、王首达、马楚天、吴昊、刘尧森、刘弘毅、刘俊邦、卢全超等学弟学妹们,作为助教,我从你们的作业和试卷中看到了许多崭新的观点,受益匪浅!

最后,感谢家人对我一贯的鼓励和支持,你们是我追求学业和未来生活的坚强后盾。没有你们的无条件支持与悉心栽培,我现在也许什么也不是。

When I just can't find my way, you are always there for me;

When I am out of my way, you are always there for me;

Thanks to all.

\vskip 18pt

\begin{flushright}

~~~~章俊彦~~~~~~~~~~

\today

\end{flushright}

\end{thanks}
