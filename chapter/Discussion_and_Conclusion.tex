\chapter{Discussion and Conclusion}

We proposed a novel method to carry out structural bionic design with GAN. Our method get rid of the reliance of large amount of expertise in traditional methods and serves as a general biomimicking process. We use the state-of-the-art algorithm in GAN (WGAN+DCGAN) to approximate real sample manifold, and show how to solve constraint problems to transfer generated image to meet users’ preference (color, shape). We also include realism and stiffness losses to judge the performance of generated image biologically (geometrically) and mechanically. The generated images are natural solutions that mix biological features, user’s design and mechanical property together. Our generated results (low resolution, missing details) are limited by WGAN and DCGAN’s performance. Our transfer results are limited by the complex and non-convex structure of the manifold. However, we believe GANs have great potential in structural bionic design, and our methods will improve as the techniques of GAN update.

